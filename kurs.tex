\documentclass{bmstu}

\begin{document}

\makecourseworktitle
{} % Название факультета
{} % Название кафедры
{Моделирование построения поверхностных и объемных геометрий с помощью операции движения} % Тема работы
{} % Номер группы/ФИО студента (если авторов несколько, их необходимо разделить запятой)
{} % ФИО научного руководителя
{} % ФИО консультанта (необязательный аргумент; если консультантов несколько, их необходимо разделить запятой)
\chapter*{ВВЕДЕНИЕ}
\addcontentsline{toc}{chapter}{ВВЕДЕНИЕ}
Для моделирование построения поверхностных и объемных геометрий необходимо использовать методы геометрического моделирования.

Поэтому, прежде, чем мы приступим к описанию построения геометрий с помощью операции движения, рассмотрим основные методы построения поверхностей и кривых, а именно  кривые и поверхности Безье, рациональные кривые и поверхности Безье и NURBS.


\chapter{Кривые и поверхности}
\section{Способы описания кривых и поверхностей}
Существует три основных подхода к описанию кривых и поверхностей.
\subsection{Явный вид}

Для кривой:
\begin{equation*}
    y=f(x), z = g(x)
\end{equation*}

Для поверхности:
\begin{equation*}
    z = f(x, y)
\end{equation*}

Этот метод имеет несколько недостатков:
\begin{itemize}
    \item Нельзя однозначно описать замкнутые кривые, например, окружности.
    \item Полученное описание не обладает инвариантностью относительно поворотов.
    \item При попытке задать кривые с очень большими углами наклона возникают большие вычислительные сложности.
\end{itemize}
\subsection{Неявные вид}

\begin{equation*}
    f(x,y,z) = 0
\end{equation*}

Недостатки:
\begin{itemize}
    \item Кривая в трёхмерном пространстве задаётся как пересечение двух поверхностей, т.е. требуется решать систему алгебраических уравнений.
    \item Сложности в процессе объединения неявно заданных фрагментов кривых
\end{itemize}

\subsection{Параметрический вид}

Параметрическое  задание кривой и поверхности преодолевает недостатки явного и неявного способов описания. С его помощью можно задавать многозначные кривые, т.е. такие зависимости, которые могут принимать несколько значений при одном значении аргумента.

Для кривой:
\begin{equation}
    \begin{cases}
        x = x(u) \\
        y = y(u) \\
        a \le u \le b
    \end{cases}
\end{equation}
и, также, будем пользоваться обозначением
\begin{equation}
    \mathbf{C}(u) = (x(u), y(u)),~        a \le u \le b
\end{equation}

Для поверхности:
\begin{equation}
    \begin{cases}
        x = x(u, v)   \\
        y = y(u, v)   \\
        z = z(u, v)   \\
        a \le u \le b \\
        c \le v \le d
    \end{cases}
\end{equation}
и, также, будем пользоваться обозначением

\begin{equation}
    \mathbf{S}(u, v) = (x(u, v), y(u, v), z(u, v)),~        a \le u \le b,~c \le v \le d
\end{equation}

\section{Кривые и поверхности Безье}

\subsection{Кривые Безье}

Пусть заданы $n+1$ точек $\mathbf{P}_i = (x_i,~y_i,~z_i)$, называемых контрольными точками. Они определяют форму и пространственное положение кривой.

Тогда кривую Безье n-ой степени можно задать с помощью уравнения:
\begin{equation}
    \mathbf{C}(u) = \sum\limits_{i=0}^n B_{i, n}(u)\mathbf{P}_i,~~~~ 0\le u\le 1б
\end{equation}
где $B_{i, n}$ - полиномы Бернштейна.
\begin{equation}
    B_{i, n}(u) = C^i_nu^i(1-u)^{n-i} = \frac{n!}{i!(n-i)!}u^i(1-u)^{n-i}
\end{equation}

Для вычисления точек кривой Безье удобно использовать алгоритм де Кастельжо:
\includelistingpretty
{deCasteljau.txt} % Имя файла с расширением (файл должен быть расположен в директории inc/lst/)
{c} % Язык программирования (необязательный аргумент)
{Псевдокод алгоритма де Кастельжо} % Подпись листинга

Например, на рисунке \ref{img:bez1} показана кривая Безье для контрольных точек $P_1 = (0,~0),~P_2=(0,~1),~P_3=(1,~2),~P_4=(3,~0)$.
\includeimage
{bez1} % Имя файла без расширения (файл должен быть расположен в директории inc/img/)
{f} % Обтекание (без обтекания)
{h} % Положение рисунка (см. figure из пакета float)
{0.5\textwidth} % Ширина рисунка
{Пример кривой Безье} % Подпись рисунка

\subsection{Поверхности Безье}

Пусть заданы контрольные точки $\mathbf{P}_{i,j}$, где $0 \le i \le n$ и $0 \le j \le m$.
Тогда поверхность Безье можно задать с помощью следующего уравнения:
\begin{equation}
    \mathbf{S}(u,v)=\sum\limits_{i=0}^n\sum\limits_{j=0}^m B_{i,n}(u)B_{j,m}(v)P_{i,j},~~~ 0\le u,v\le 1
\end{equation}
Аналогично кривым Безье, точки поверхности Безье можно находить с помощью алгоритма де Кастельжо из листинга \ref{lst:deCasteljau.txt}.
\includelistingpretty
{deCasteljau2.txt} % Имя файла с расширением (файл должен быть расположен в директории inc/lst/)
{c} % Язык программирования (необязательный аргумент)
{Псевдокод алгоритма де Кастельжо для поверхности} % Подпись листинга
На рисунке \ref{img:bez2} показан пример поверхности Безье для 15 контрольных точек.

\includeimage
{bez2} % Имя файла без расширения (файл должен быть расположен в директории inc/img/)
{f} % Обтекание (без обтекания)
{h} % Положение рисунка (см. figure из пакета float)
{0.5\textwidth} % Ширина рисунка
{Пример поверхности Безье} % Подпись рисунка

\end{document}