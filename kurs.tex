\documentclass{bmstu}
\usepackage{tocloft}
\renewcommand{\cftpartleader}{\cftdotfill{\cftdotsep}}
\renewcommand{\cftchapleader}{\cftdotfill{\cftdotsep}}

%нужные мне штучки
\usepackage{relsize}
\newcommand{\Christoffel}{\ensuremath{\mathlarger{\mathlarger\Gamma}}}

\usepackage{stackengine,scalerel}
\newcommand\overstarbf[1]{\ThisStyle{\ensurestackMath{%
			\stackengine{0pt}{\SavedStyle\mathbf{#1}}{\smash{\SavedStyle*}}{O}{c}{F}{T}{S}}}}
\newcommand\overstar[1]{\ThisStyle{\ensurestackMath{%
			\setbox0=\hbox{$\SavedStyle#1$}%
			\stackengine{0pt}{\copy0}{\kern.2\ht0\smash{\SavedStyle*}}{O}{c}{F}{T}{S}}}}


\bibliography{bib}

\nocite{*}

\newenvironment{gost-itemize}
{\begin{itemize}[label=---,itemindent=\parindent,leftmargin=0pt]}
	{\end{itemize}}

    \renewcommand{\maketableofcontents}
    {
        \setlength{\cftbeforetoctitleskip}{-2em}
        \setlength{\cftaftertoctitleskip}{0em}
        \renewcommand\contentsname{
            \normalsize \centerline{\MakeUppercase{Содержание}}\hfill c.
        }
        %\singlespacing
        \begin{singlespace}
            \tableofcontents
        \end{singlespace}
    }
    
\begin{document}

\makecourseworktitle
{} % Название факультета
{} % Название кафедры
{Моделирование построения поверхностных и объемных геометрий с помощью операции движения} % Тема работы
{} % Номер группы/ФИО студента (если авторов несколько, их необходимо разделить запятой)
{} % ФИО научного руководителя
{} % ФИО консультанта (необязательный аргумент; если консультантов несколько, их необходимо разделить запятой)

\maketableofcontents

\chapter*{ВВЕДЕНИЕ}
\addcontentsline{toc}{chapter}{ВВЕДЕНИЕ}
Курсовая работа – вид учебной работы, направленный на развитие практических навыков и умений, а также формирование компетенций обучающихся в процессе выполнения определенных видов заданий, связанных с будущей профессиональной деятельностью. Работа рассчитана на закрепление и применение полученных навыков в процессе учёбы.

Целью данной курсовой работы является решение заданных задач. Задачами этой работы является (с полным текстом задания можно ознакомиться в ПРИЛОЖЕНИИ Б):
\begin{gost-itemize}
\item разработать формулы и алгоритмы, описывающие геометрии различных тел движения
\item получить визуализацию тел движения, реализовать возможность изменения их формы.
\end{gost-itemize}

Для моделирование построения поверхностных и объемных геометрий необходимо использовать методы геометрического моделирования.
Поэтому в курсовой работе будут рассмотрены основные методы построения поверхностей и кривых, а именно  кривые и поверхности Безье, рациональные кривые и поверхности Безье, B-Spline и NURBS.

Затем будут рассмотрены методы построения поверхностей выдавливания, вращения, сдвига, а также кинематических поверхностей.

В конце будет рассмотрена программная реализация изложенных методов, поддерживающая визуализацию полученных поверхностей и кривых и их изменение в реальном времени.



\chapter{Кривые и поверхности}
\section{Способы описания кривых и поверхностей}
Существует три основных подхода к описанию кривых и поверхностей.
\subsection{Явный вид}

Для кривой:
\begin{equation*}
    y=f(x), z = g(x)
\end{equation*}

Для поверхности:
\begin{equation*}
    z = f(x, y)
\end{equation*}

Этот метод имеет несколько недостатков:
\begin{gost-itemize}
    \item Нельзя однозначно описать замкнутые кривые, например, окружности.
    \item Полученное описание не обладает инвариантностью относительно поворотов.
    \item При попытке задать кривые с очень большими углами наклона возникают большие вычислительные сложности.
\end{gost-itemize}
\subsection{Неявные вид}

\begin{equation*}
    f(x,y,z) = 0
\end{equation*}

Недостатки:
\begin{gost-itemize}
    \item Кривая в трёхмерном пространстве задаётся как пересечение двух поверхностей, т.е. требуется решать систему алгебраических уравнений.
    \item Сложности в процессе объединения неявно заданных фрагментов кривых
\end{gost-itemize}

\subsection{Параметрический вид}

Параметрическое  задание кривой и поверхности преодолевает недостатки явного и неявного способов описания. С его помощью можно задавать многозначные кривые, т.е. такие зависимости, которые могут принимать несколько значений при одном значении аргумента.

Для кривой:
\begin{equation}
    \begin{cases}
        x = x(u) \\
        y = y(u) \\
        a \le u \le b
    \end{cases}
\end{equation}
и также будем пользоваться обозначением
\begin{equation}
    \mathbf{C}(u) = (x(u), y(u)),~        a \le u \le b
\end{equation}

Для поверхности:
\begin{equation}
    \begin{cases}
        x = x(u, v)   \\
        y = y(u, v)   \\
        z = z(u, v)   \\
        a \le u \le b \\
        c \le v \le d
    \end{cases}
\end{equation}
и также будем пользоваться обозначением

\begin{equation}
    \mathbf{S}(u, v) = (x(u, v), y(u, v), z(u, v)),~        a \le u \le b,~c \le v \le d
\end{equation}

\section{Кривые и поверхности Безье}

\subsection{Кривые Безье}

Пусть заданы $n+1$ точек $\mathbf{P}_i = (x_i,~y_i,~z_i)$, называемых \textit{контрольными точками}. Они определяют форму и пространственное положение кривой.

Тогда \textit{кривую Безье n-ой степени} можно задать с помощью уравнения:
\begin{equation}
    \mathbf{C}(u) = \sum\limits_{i=0}^n B_{i, n}(u)\mathbf{P}_i,~~~~ 0\le u\le 1б
\end{equation}
где $B_{i, n}$ - полиномы Бернштейна.
\begin{equation}
    B_{i, n}(u) = C^i_nu^i(1-u)^{n-i} = \frac{n!}{i!(n-i)!}u^i(1-u)^{n-i}
\end{equation}

Для вычисления точек кривой Безье удобно использовать алгоритм де Кастельжо:
\includelistingpretty
{deCasteljau.txt} % Имя файла с расширением (файл должен быть расположен в директории inc/lst/)
{c} % Язык программирования (необязательный аргумент)
{Псевдокод алгоритма де Кастельжо} % Подпись листинга

Например, на Рисунке \ref{img:bez1} показана кривая Безье для контрольных точек $P_1 = (0,~0),~P_2=(0,~1),~P_3=(1,~2),~P_4=(3,~0)$.
\includeimage
{bez1} % Имя файла без расширения (файл должен быть расположен в директории inc/img/)
{f} % Обтекание (без обтекания)
{h} % Положение рисунка (см. figure из пакета float)
{0.5\textwidth} % Ширина рисунка
{Пример кривой Безье} % Подпись рисунка

\subsection{Поверхности Безье}

Пусть заданы контрольные точки $\mathbf{P}_{i,j}$, где $0 \le i \le n$ и $0 \le j \le m$.
Тогда \textit{поверхность Безье} можно задать с помощью следующего уравнения:
\begin{equation}
    \mathbf{S}(u,v)=\sum\limits_{i=0}^n\sum\limits_{j=0}^m B_{i,n}(u)B_{j,m}(v)P_{i,j},~~~ 0\le u,v\le 1
\end{equation}
Аналогично кривым Безье, точки поверхности Безье можно находить с помощью алгоритма де Кастельжо из Листинга \ref{lst:deCasteljau.txt}.
\includelistingpretty
{deCasteljau2.txt} % Имя файла с расширением (файл должен быть расположен в директории inc/lst/)
{c} % Язык программирования (необязательный аргумент)
{Псевдокод алгоритма де Кастельжо для поверхности} % Подпись листинга
На Рисунке \ref{img:bez2} показан пример поверхности Безье для 15 контрольных точек.

\includeimage
{bez2} % Имя файла без расширения (файл должен быть расположен в директории inc/img/)
{f} % Обтекание (без обтекания)
{h} % Положение рисунка (см. figure из пакета float)
{0.5\textwidth} % Ширина рисунка
{Пример поверхности Безье} % Подпись рисунка

\section{Рациональные кривые и поверхности Безье}
\subsection{Рациональные кривые Безье}

Так как кривые Безье - полиномиальные кривые, они имеют существенный недостаток, а именно с их помощью невозможно задать некоторые виды кривых, такие как окружности, эллипсы, гиперболы и прочие. Данные виды кривых можно задать с помощью рациональных функций, то есть как частное двух полиномов.
\begin{equation}\label{rational_func}
    x(u) = \frac{X(u)}{W(u)}~~~~y(u) = \frac{Y(u)}{W(u)},
\end{equation}

где $X(u)$, $Y(u)$ и $W(u)$ - полиномы.

Заметим также, что каждая координатная функция имеет одинаковый знаменатель $W(u)$.

Рациональные кривые с координатными функциями в виде (\ref{rational_func}) имеют элегантную геометрическую интерпретацию, которая дает эффективные методы построения этих кривых и небольшие требования к памяти компьютера.

Оказывается, что можно использовать однородные координаты, чтобы задать рациональные кривые в $n$-мерном пространстве с помощью полиномиальной кривой в $(n+1)$-мерном пространстве.

Рассмотрим точку в евклидовом пространстве $\mathbf{P} = (x,y,z)$. Затем запишем точку $\mathbf{P}$ как $\mathbf{P^\omega}=(\omega x, \omega y, \omega z, \omega)=(X, Y, Z, W)$ в четырех-мерном пространстве, причем $\omega \neq 0$. Тогда $\mathbf{P}$ можно получить из $\mathbf{P^\omega}$ делением всех координат на четвертую координату $W$, то есть с помощью отображения $P^\omega$ на гиперплоскость $W=1$
\includeimage
{proj} % Имя файла без расширения (файл должен быть расположен в директории inc/img/)
{f} % Обтекание (без обтекания)
{h} % Положение рисунка (см. figure из пакета float)
{0.5\textwidth} % Ширина рисунка
{Представление точки евклидова пространства в однородной форме для двумерного случая} % Подпись рисунка

Данное отображение $H$ является перспективной проекцией с центром в начале координат:
\begin{equation}\label{projfunc}
    \mathbf{P} = H\{\mathbf{P^\omega}\} = H\{(X, Y, Z, W)\} =
    \left(\frac{X}{W}, \frac{Y}{W}, \frac{Z}{W}\right)
\end{equation}

Тогда для множества контрольных точек $\{\mathbf{P_i}\}$ и множества весов $\{\omega_i\}$ зададим множество взвешенных контрольных точек $\mathbf{P}_i^\omega=(\omega_ix_i,\omega_iy_i,\omega_iz_i, \omega_i)$. Тогда нерациональная (полиномиальная) кривая Безье в 4-х мерном пространстве
\begin{equation}\label{4bez}
    \mathbf{C}^\omega(u) = \sum\limits_{i=0}^nB_{i,n}(u)\mathbf{P}_i^\omega
\end{equation}

Уравнение (\ref{4bez}) в координатном виде:
\begin{equation*}
    X(u) = \sum\limits_{i=0}^nB_{i,n}(u)\omega_ix_i~~~~~~ Y(u) = \sum\limits_{i=0}^nB_{i,n}(u)\omega_iy_i
\end{equation*}
\begin{equation*}
    Z(u) = \sum\limits_{i=0}^nB_{i,n}(u)\omega_iz_i~~~~~~ W(u) = \sum\limits_{i=0}^nB_{i,n}(u)\omega_i
\end{equation*}

Заметим, что $W \neq 0$ поскольку мы выбираем $\omega_i > 0$.

Применяя к (\ref{4bez}) отображение (\ref{projfunc}), получим искомую рациональную кривую Безье в 3-х мерном пространстве, задающуюся формулами
\[
    x(u) = \frac{X(u)}{W(u)}= \frac{\sum\limits_{i=0}^nB_{i,n}(u)\omega_ix_i}{\sum\limits_{i=0}^nB_{i,n}(u)\omega_i}
\]
\[
    y(u) = \frac{Y(u)}{W(u)}= \frac{\sum\limits_{i=0}^nB_{i,n}(u)\omega_iy_i}{\sum\limits_{i=0}^nB_{i,n}(u)\omega_i}
\]
\[
    z(u) = \frac{Z(u)}{W(u)}= \frac{\sum\limits_{i=0}^nB_{i,n}(u)\omega_iz_i}{\sum\limits_{i=0}^nB_{i,n}(u)\omega_i}
\]
или в векторной записи
\begin{equation}\label{ratcurve}
    \mathbf{C}(u) = \frac{\sum\limits_{i=0}^nB_{i,n}(u)\omega_i\mathbf{P}_i}{\sum\limits_{i=0}^nB_{i,n}(u)\omega_i}
\end{equation}



Например, если взять $\mathbf{P}_0 = (1,0)$, $\mathbf{P}_1 = (1,1)$, $\mathbf{P}_2 = (0,1)$ и $\omega_i = (1, 1, 2)$, получим дугу окружности (Рисунок \ref{img:arc}).
\includeimage
{arc} % Имя файла без расширения (файл должен быть расположен в директории inc/img/)
{f} % Обтекание (без обтекания)
{h} % Положение рисунка (см. figure из пакета float)
{0.5\textwidth} % Ширина рисунка
{Дуга окружности, построенная с помощью рациональной кривой Безье} % Подпись рисунка

Если веса всех вершин равны, то получим обычную кривую Безье, поскольку в таком случае знаменатель в уравнение (\ref{ratcurve}) - это просто сумма полиномов Бернштейна, которая равна 1. Таким образом, рациональные кривые Безье являются обобщением полиномиальных кривых Безье.
\subsection{Рациональные поверхности Безье}

Аналогично рациональным кривым Безье, рациональные поверхности Безье можно представить как перспективную проекцию 4-х мерной полиномиальной поверхности Безье
\[
    \mathbf{S}^\omega(u, v) = \sum\limits_{i=0}^n\sum\limits_{j=0}^mB_{i,n}(u)B_{j,m}(v)\mathbf{P}^\omega_{i,j}
\]
\begin{multline}
    \mathbf{S}(u,v) = H\{\mathbf{S}^\omega(u, v)\} =  \frac{\sum\limits_{i=0}^n\sum\limits_{j=0}^mB_{i,n}(u)B_{j,m}(v)\omega_{i, j}\mathbf{P}_{i,j}}{\sum\limits_{i=0}^n\sum\limits_{j=0}^mB_{i,n}(u)B_{j,m}(v)\omega_{i, j}} =\\= \sum\limits_{i=0}^n\sum\limits_{j=0}^mR_{i,j}(u,v)\mathbf{P}_{i,j},
\end{multline}
где
\begin{equation}
    R_{i,j}(u, v) = \frac{B_{i, n}(u)B_{j, m}(v)}{\sum\limits_{r=0}^n\sum\limits_{s=0}^mB_{r,n}(u)B_{s,m}(v)\omega_{r, s}}
\end{equation}

На Рисунке \ref{img:arcsur} изображена цилиндрическая поверхность, построенная с помощью рациональной поверхности Безье. Она представляет собой поверхность, полученную движением дуги окружности из Рисунка \ref{img:arc}.
\includeimage
{arcsur} % Имя файла без расширения (файл должен быть расположен в директории inc/img/)
{f} % Обтекание (без обтекания)
{h} % Положение рисунка (см. figure из пакета float)
{0.5\textwidth} % Ширина рисунка
{Цилиндрическая поверхность, построенная с помощью рациональной поверхности Безье} % Подпись рисунка

\section{B-сплайны}

\subsection{B-сплайн кривая}

У кривых заданных полиномами или рациональными функциями есть несколько минусов.
\begin{gost-itemize}
    \item Для большого числа точек требуется полиномы большой степени. Так для того, чтобы построить кривую через $n$ точек, требуется полином $n-1$ степени. Кривые, заданные полиномами с большими степенями, тяжело обрабатывать, а также они численно неустойчивы.
    \item Для сложных кривых также требуется большая степень полинома.
    \item Полиномиальные кривые не очень подходят для проектирования кривой. Хотя в кривых Безье и можно менять форму кривой, изменяя контрольные точки и  значения весов в них, кривая меняется нелокально, т.е. изменение параметров одной точки меняет всю кривую.
\end{gost-itemize}

B-сплайны лишены этих недостатков: степень полинома B-сплайна можно задать независимо от числа контрольных точек, а также они B-сплайны допускают локальный контроль над формой кривой.

Поставим задачу следующим образом. Пусть даны контрольные точки $\mathbf{P}_i$. Определим кривую по формуле
\begin{equation}
    C(u) = \sum\limits_{i=0}^{n}N_i(u)\mathbf{P_i},~~~~u_{min}\le u \le u_{max}
\end{equation}
где $N_i(u)$ - набор кусочно-полиномиальных функций, таких, что
\begin{enumerate}
    \item $N_i(u) = 0$ при $u\notin[a_i, b_i]\subset[u_{min}, u_{max}]$;
    \item $N_i(u)$ линейно независимы и образуют базис;
    \item $\sum\limits_{i=0}^nN_i(u) = 1$ для каждого $u\in[u_{min}, u_{max}]$.
\end{enumerate}

Решение поставленной задачи даётся \textit{B-сплайнами} (сокр. от basis).

Общее выражение для расчёта координат точек B-сплайна:
\begin{equation}
    C(u) = \sum\limits_{i=0}^nN_{i,p}(u)\mathbf{P}_i
\end{equation}

В 1972 году Кокс и де Бур предложили использовать функции $N_{i, p}$, определяемые рекурсивно. Пусть $U=\{u_0,\dots,u_m\}$ - неубывающая последовательно  вещественных чисел, т.е. $u_i\le u_{i+1}, i=0,\dots,m-1$. $u_i$ называют \textit{узлами} (knot), а $U$ - \textit{вектором узлов} (knot vector). Тогда \textit{$i$-тая базисная функция B-сплайна $p$-ой степени}, обозначаемая $N_{i,p}(u)$, выражается следующим образом:
\begin{equation}
    N_{i, 0}(u) = \begin{cases}
        1, u\in[u_{i}, u_{i+1}] \\
        0, u\notin[u_{i}, u_{i+1}]
    \end{cases}
\end{equation}
\begin{equation}
    N_{i, p}(u) = \frac{u-u_i}{u_{i+p}-u_i}N_{i, p-1}(u)+\frac{u_{i+p+1}-u}{u_{i+p+1}-u_{i+1}}N_{i+1, p-1}(u)
\end{equation}

В силу свойств базисных функций B-сплайна в любом заданном промежутке $[u_i, u_{i+1}]$ могут быть отличны от нуля только $p+1$ функций: $N_{i-p, p},\dots,N_{i,p}$. Например, единственные кубические функции, отличные от нуля на $[u_3, u_4]$ - это функции $N_0^3,\dots,N_3^3$.
Поэтому при вычислении базисных функций $N^k$ в точке $u$, важно уметь находить индекс $i$ в векторе узлов, при котором выполняется соотношение $u_i\le u \le u_{i+1}$.
Для этого например можно использовать алгоритм бинарного поиска:
\includelistingpretty
{binary.txt} % Имя файла с расширением (файл должен быть расположен в директории inc/lst/)
{c} % Язык программирования (необязательный аргумент)
{Алгоритм бинарного поиска индекса $i$} % Подпись листинга

Также, по этой же причине следует, что чтобы находить значение сплайна для любого $u$ из отрезка $[u_{i}, u_{i+1}]$, необходимо иметь не менее $p$ дополнительных узлов до и после него. На практике этого обычно достигают, дублируя первый и последний узел нужное число раз. Например, если $p = 3$ и даны узлы в точках $\{ 0 , 1 , 2 \}$, то расширенный массив узлов будет иметь вид: $\{ 0 , 0 , 0 , 0 , 1 , 2 , 2 , 2 , 2\}$.

Для вычисления ненулевых базисных функций $N_{i-p, p},\dots,N_{i,p}$ можно использовать следующий алгоритм:
\includelistingpretty
{bspl.txt} % Имя файла с расширением (файл должен быть расположен в директории inc/lst/)
{c} % Язык программирования (необязательный аргумент)
{Алгоритм вычисления базисных функций $N_{i-p, p},\dots,N_{i,p}$} % Подпись листинга


Также следует заметить, что существуют разные подходы к заданию узлового вектора. Разные методы задания узловых значений позволяют получить разные
функции сопряжения и, соответственно, разные кривые. Если расстояние
между значениями в узлах постоянно, получающаяся в результате кривая
называется \textit{равномерным B-сплайном}. Например, можно задать
следующий равномерный вектор узлов:
\[
  \{-1.5, -1.0, -0.5, 0.0, 0.5, 1.0, 1.5, 2.0\}  
\]

Часто значения узлов нормируются в диапозон от $0$ до $1$.

Аналогично, если допускается выбор одинаковых внутренних значений узлов и
неравномерное размещение значений узлов, то такое B-сплайн называется \textit{неравномерным}.

Так, например, для контрольных точек $\mathbf{P}_0 = (0,0), \mathbf{P}_1 = (0,1), \mathbf{P}_2 = (1,2), \mathbf{P}_3 = (3,0)$ и узлового вектора $U=\{0,0,0,\frac{1}4,1,1,1\}$ B-сплайн 2-степени будет иметь вид:
\includeimage
{bspl1} % Имя файла без расширения (файл должен быть расположен в директории inc/img/)
{f} % Обтекание (без обтекания)
{h} % Положение рисунка (см. figure из пакета float)
{0.5\textwidth} % Ширина рисунка
{Пример B-сплайн кривой} % Подпись рисунка

\subsection{Свойства B-сплайна}
B-сплайн имеет следующие свойства:
\begin{gost-itemize}
    \item полиномиальная кривая имеет степень $p$ и непрерывность $C^{p-1}$;
    \item диапазон параметра $u$ делится на $n + p + 1 $ подынтервалов $n + p + 2$ значениями, заданными в векторе узлов;
    \item если значения узлов обозначить $\{u_0, u_1,\dots, u_{n+p+1}\}$, получающийся B-сплайн определяется только в промежутке $[u_p, u_{n+1})$, т.к. только в этом промежутке $\sum\limits_{i=0}^nN_{i,p}=1$;
    \item каждый участок сплайна определяется $p + 1$ контрольными точками;
    \item локальная коррекция: любая контрольная точка $\mathbf{P}_i$ может влиять на форму кривой $\mathbf{C}(u)$ только на интервале $[u_i, u_{i+p+1})$;
    \item при движении вдоль кривой, функции $N_{i,p}$ действуют подобно
    переключателям. Когда $u$ проходит мимо узла $u_{i+p+1}$ в векторе узлов,
    функция $N_{i,p}$ (и, соответственно, точка $\mathbf{P}_i$) выключаются, поскольку становится равной нулю, и включаются следующие;
    \item чем меньше степень кривой, тем ближе она подходит к контрольным
    точкам. Кривые высоких порядков более гладкие;
    \item помимо локального контроля B-сплайны позволяют варьировать число
    контрольных точек, используемых в разработке кривой, без изменения
    степени полинома;
    \item кривые на базе В-сплайнов аффинно инвариантны. Для
    преобразования В-сплайн кривой мы просто преобразуем каждую
    контрольную точку и генерируем новую кривую;
    \item В-сплайн кривая является выпуклой комбинацией своих контрольных
    точек и поэтому лежит внутри их выпуклой оболочки. Возможно
    более сильное утверждение: при любом значении $u \in [u_p, u_{n+1}]$ только
    $p + 1$ функций В-сплайна «активны» (то есть отличны от нуля). В
    этом случае кривая должна лежать внутри выпуклой оболочки не
    более $p + 1$ последовательных активных контрольных точек;
    \item В-сплайн кривые обеспечивают линейную точность: если $p + 1$
последовательных контрольных точек коллинеарны, то их выпуклая
оболочка будет прямой линией, и кривая будет захвачена внутрь её;
\item В-сплайн кривые уменьшают колебания: В-сплайн кривая не
пересекает никакую линию чаще, чем её контрольный полигон.
\end{gost-itemize}

\subsection{Призводные B-сплайн кривой}

Обозначим за $\mathbf{C}^{k}(u)$ $k$-ую производную кривой $\mathbf{C}(u)$. Если зафиксировать $u$, тогда мы можем получить $\mathbf{C}^{k}(u)$ вычисляя $k$-ую производную базисных функций с помощью следующих формул:
\begin{equation}
    N'_{i,p}=\frac{p}{u_{i+p}-u_i}N_{i, p-1}(u) - \frac{p}{u_{i+p+1}-u_{i+1}}N_{i+1, p-1}(u)
\end{equation}
\begin{equation}
    N^{(k)}_{i,p}=p\left(\frac{N^{(k-1)}_{i, p-1}}{u_{i+p}-u_i}-\frac{N^{(k-1)}_{i+1, p-1}}{u_{i+p+1}-u_{i+1}}\right)
\end{equation}

А также можно использовать алгоритм, приведенный в Листинге \ref{lst:ders.txt} в ПРИЛОЖЕНИИ А.

Тогда $k$-ая производная кривой $\mathbf{C}(u)$
\begin{equation}\label{BSder}
    \mathbf{C}^{k}(u) = \sum\limits_{i=0}^{n}N^{(k)}_{i,p}\mathbf{P}_i,
\end{equation}
которую можно вычислить с помощью следующего алгоритма:
\includelistingpretty
{curveder.txt} % Имя файла с расширением (файл должен быть расположен в директории inc/lst/)
{c++} % Язык программирования (необязательный аргумент)
{Алгоритм вычисления производных кривой} % Подпись листинга

\subsection{B-сплайн поверхности}

B-сплайн поверхность задается с помощью контрольных точек и двух векторов узлов. Ее точки можно найти с помощью формулы:
\begin{equation}
    S(u, v) = \sum\limits_{i=0}^n\sum\limits_{j=0}^m N_{i,p}(u)N_{j,q}(v)\mathbf{P_{i,j}}
\end{equation}
Например, для набора контрольных точек
\begin{equation*}
    \begin{matrix}

        \mathbf{P_{0,0}}=(0,0,0),     &  & \mathbf{P_{0,1}}=(1,0,0),       &  & \mathbf{P_{0,2}}=(2,0,0),     \\
        \mathbf{P_{1,0}}=(0,0.5,1.3), &  & \mathbf{P_{1,1}}=(1,0.5,1.2)  , &  & \mathbf{P_{1,2}}=(2,0.5,1.3), \\
        \mathbf{P_{2,0}}=(0,1,0)    , &  & \mathbf{P_{2,1}}=(1,1,0)      , &  & \mathbf{P_{2,2}}=(2,1,0)    , \\
        \mathbf{P_{3,0}}=(0,1.5,1.3), &  & \mathbf{P_{3,1}}=(1,1.5,1.2)  , &  & \mathbf{P_{3,2}}=(2,1.5,1.3), \\
        \mathbf{P_{4,0}}=(0,2,0)    , &  & \mathbf{P_{4,1}}=(1,2,0)      , &  & \mathbf{P_{4,2}}=(2,2,0)    ,
    \end{matrix}
\end{equation*}
векторов узлов $U = \{0,0,0,1/2,1/2,1,1,1\}$, $V = \{0,0,0,1,1,1\}$ и $p=q=2$ получим следующую B-сплайн поверхность:
\includeimage
{bspl2} % Имя файла без расширения (файл должен быть расположен в директории inc/img/)
{f} % Обтекание (без обтекания)
{h} % Положение рисунка (см. figure из пакета float)
{0.5\textwidth} % Ширина рисунка
{Пример B-сплайн поверхности} % Подпись рисунка
% \newpage
\section{NURBS}
\subsection{Рациональный B-сплайн и NURBS}
Аналогично случаю рациональных кривых Безье, контрольные точки
рационального В-сплайна указываются с использованием однородных
координат. Функции сопряжения применяются именно к этим однородным
координатам. Координаты точки рационального B-сплайна в однородном
пространстве получаются по формулам:

\begin{equation*}
    X(u) = \sum\limits_{i=0}^nN_{i,p}(u)\omega_ix_i~~~~~~ Y(u) = \sum\limits_{i=0}^nN_{i,p}(u)\omega_iy_i
\end{equation*}
\begin{equation*}
    Z(u) = \sum\limits_{i=0}^nN_{i,p}(u)\omega_iz_i~~~~~~ W(u) = \sum\limits_{i=0}^nN_{i,p}(u)\omega_i
\end{equation*}

Тогда уравнение рационального B-сплайна в трехмерном пространстве в вектором виде примет вид:
\begin{equation}\label{ratcurve}
    \mathbf{C}(u) = \frac{\sum\limits_{i=0}^nN_{i,p}(u)\omega_i\mathbf{P}_i}{\sum\limits_{i=0}^nN_{i,p}(u)\omega_i}=\sum\limits_{i=0}^nR_{i,p}\mathbf{P}_i,
\end{equation}
где
\begin{equation*}
    R_{i,p} = \frac{N_{i,p}(u)\omega_i}{\sum\limits_{j=0}^nN_{j,p}(u)\omega_j}
\end{equation*}
 --- базисные функции рационального B-сплайна.

Рациональные B-сплайны и их базисы это обобщение нерациональных
B-сплайнов и базисов. При $\omega_i \ge 0$ для всех $i$ они наследуют почти все
аналитические и геометрические свойства последних. В частности:
\begin{gost-itemize}
    \item каждая функция рационального базиса положительна или равна нулю для всех значений параметра, т.е. $R_{i,p}\ge0$;
    \item при $p > 0$ каждая функция $R_{i, p}(u)$ имеет ровно один максимум;
    \item рациональный B-сплайн степени $p$ имеет непрерывность $C^{p-1}$;
    \item максимальная степень рационального B-сплайна равна количеству контрольных точек минус 1;
    \item если $u\in[u_i, u_{i+1})$, то $\mathbf{C}(u)$ находится в пределах выпуклой оболочки, составленной из контрольных точек $\mathbf{P}_{i-p},\dots,\mathbf{P}_{i}$;
    \item свойство локальности $R_{i,p}(u)=0$ для $u\notin[u_i, u_{i+p+1})$. Если $u\in[u_i, u_{i+1})$, только функции $R_{i-p, p},\dots,R_{i,p}$ являются ненулевыми;
    \item аффинная и перспективная инвариантность: применяемое к кривой
    преобразование можно свести к преобразованию только ее
    контрольных точек;
    \item свойство уменьшения вариации: прямая или плоскость пересекают
    сплайн не большее количество раз чем контрольный полигон сплайна.
\end{gost-itemize}

Также с помощью квадратичных рациональных B-сплайнов можно целиком построить окружность или какое-либо другое коническое сечение. То есть с их помощью можно сшить отдельные дуги, представляемые с помощью рациональных сплайнов Безье. 

Окружность можно задать множеством различных способов, например, задав в вершинах и на серединах сторон девять контрольных точек, причём начальная и конечная вершины должны совпадать в одной из середин (Рисунок \ref{img:circle}). Узловой вектор можно задать в следующем виде: $\{0,0,0,\frac14, \frac14,\frac12,\frac12,\frac34,\frac34,1,1,1\}$. Вес контрольной точки $\omega_i=1$, если $i$ --- четное и $\omega_i=\frac{\sqrt2}2$, если $i$ --- нечетное.
\includeimage
{circle} % Имя файла без расширения (файл должен быть расположен в директории inc/img/)
{f} % Обтекание (без обтекания)
{h} % Положение рисунка (см. figure из пакета float)
{0.5\textwidth} % Ширина рисунка
{Способ задания контрольных точек для окружности} % Подпись рисунка

Обычно в пакетах графической разработки для построения рациональных
B-сплайнов используются неравномерные представления вектора узлов.
Данные сплайны называются \textit{«NURBS»} (Nonuniform Rational B-splines — неравномерные рациональные B-сплайны). NURBS с 1983 г. являются стандартом IGES. IGES --- это стандарт обмена проектной информацией между системами автоматизированного проектирования, а также между ними и системами автоматизированного производства.

\subsection{Производная NURBS-кривой}

Выше были приведены формулы и алгоритмы для вычисления производных B-сплайн кривой. Эти формулы можно применять и к $\mathbf{C}^\omega(u)$, поскольку это нерациональная кривая в четырехмерном пространстве. Таким образом, производные кривой $\mathbf{C}(u)$ можно выразить через производные кривой $\mathbf{C}^\omega(u)$.
Пусть
\begin{equation}
    \mathbf{C}(u) = \frac{\omega(u)\mathbf{C}(u)}{\omega(u)} = \frac{\mathbf{A}(u)}{\omega(u)},
\end{equation}
где $\mathbf{A}(u)$ - векторная функция, координаты которой являются первыми тремя координатами функции $\mathbf{C}^\omega(u)$.
Тогда
\begin{multline}
    \mathbf{C}'(u)=\frac{\omega(u)\mathbf{A}'(u)-\omega'(u)\mathbf{A}(u)}{\omega(u)^2}=\\=\frac{\omega(u)\mathbf{A}'(u)-\omega'(u)\omega(u)\mathbf{C}(u)}{\omega(u)^2}=\frac{\mathbf{A}'(u)-\omega'(u)\mathbf{C}(u)}{\omega(u)}
\end{multline}

Так как $\mathbf{A}(u)$ и $\omega(u)$ представляют собой координаты $\mathbf{C}^\omega(u)$, можно получить первые производные используя уравнение (\ref{BSder}). Последующие производные можно получить, дифференцируя функцию $\mathbf{A}$, используя правило Лейбница:
\begin{multline}
    \mathbf{A}^{(k)}(u)=(\omega(u)\mathbf{C}(u))^{(k)}=\sum\limits_{i=0}^k\binom{k}{i}\omega^{(i)}(u)\mathbf{C}^{(k-i)}(u)=\\=\omega(u)\mathbf{C}^{(k)}(u)+\sum\limits_{i=1}^{k}\binom{k}{i}\omega^{(i)}(u)\mathbf{C}^{(k-i)}(u)
\end{multline}
откуда получаем
\begin{equation}
    \mathbf{C}^{(k)}(u) = \frac{\mathbf{A}^{(k)}(u)-\sum\limits_{i=1}^{k}\binom{k}{i}\omega^{(i)}(u)\mathbf{C}^{(k-i)}(u)}{\omega(u)}
\end{equation}

Производные $\mathbf{A}^{(k)}(u)$ и $\omega^{(i)}(u)$ могут быть получены с помощью алгоритма, приведенного в Листинге \ref{lst:curveder.txt}.

Теперь, если $u$ зафиксировано, а производные от нулевой до $d$ функий $\mathbf{A}(u)$ и $\omega(u)$ вычислены и загружены в массивы Aders и wders соответственно, т.е. $\mathbf{C}^\omega(u)$ продифференцировано и его координаты разделены на массивы Aders и wders, то с помощью алгоритма, приведенного в Листинге \ref{lst:ratcurvder.txt}, можно вычислить точку кривой и все производные в ней $\mathbf{C}^{(k)}(u),~ 1 \le k \le d$.
\includelistingpretty
{ratcurvder.txt} % Имя файла с расширением (файл должен быть расположен в директории inc/lst/)
{c} % Язык программирования (необязательный аргумент)
{Алгоритм вычисления производных рациональной B-сплайн кривой} % Подпись листинга


\subsection{NURBS-поверхности}

Для задания NURBS-поверхности будем использовать следующую формулу:
\begin{equation}
    \mathbf{S}(u, v) = \frac{\sum\limits_{i=0}^n\sum\limits_{j=0}^m\omega_{ij}N_{i,p}(u)N_{j,q}(v)\mathbf{P_{i,j}}}{\sum\limits_{i=0}^n\sum\limits_{j=0}^m\omega_{ij}N_{i,p}(u)N_{j,q}(v)},
\end{equation}
где $\mathbf{P_{ij}}$ --- контрольные точки, $\omega_{ij}$ --- их веса.

Благодаря общности и гибкости NURBS-поверхности стали пользоваться
популярностью. Поскольку В-сплайны являются частным случаем
NURBS-поверхностей (при $\omega_{ij} = 1$), можно использовать единый алгоритм для создания обширного семейства поверхностей. NURBS
поверхности обладают большинством свойств, присущих B-сплайновым
поверхностям и NURBS-кривым. NURBS поверхности позволяют точно
описывать квадратичные поверхности, такие как цилиндр, конус, сфера,
параболоид и гиперболоид. Поэтому дизайнеру вместо инструментария,
состоящего из большого числа различных алгоритмов для создания
поверхностей, потребуется всего один метод.
\chapter{Поверхности движения}

Многие модели или заготовки для них можно получить с помощью
заметания (sweeping), т.е. путём движения кривой по заданной
траектории. Такие объекты обладают трансляционной, вращательной или
другой симметрией. Пусть траектория движения описывается кривой $\mathbf{g}(v)$, которую будем называть \textit{направляющей}. Движущуюся по траектории кривую линию будем называть \textit{образующей кривой}. Направляющая кривая и образующая кривая не должны иметь точек самопересечения. Набор таких двумерных примитивов, как окружности и прямоугольники, может предлагаться в качестве образующих как пункты меню. Существуют и другие методы получения двумерных фигур, например, построение замкнутых сплайновых кривых.
Если образующая кривая не замкнута, то на её основе в общем случае
нельзя построить тело. Обычно из незамкнутой кривой создаётся
замкнутая составная кривая путём «придания ей толщины» с помощью
эквидистантных кривых. В общем случае образующая представляет собой
замкнутую составную фигуру. Если образующая является плоской кривой,
то можно построить тело с плоскими торцами.
В популярной системе трёхмерной графики 3Ds Max заметание называется
Loft, в описаниях пакета на русском языке его именуют лофтингом.
\section{Поверхность выдавливания}

Пусть задана некоторая кривая $\mathbf{C}(u)$ и единичный вектора $\mathbf{d}$. Если направляющей движения контура служит отрезок прямой $\mathbf{g}(v) = \mathbf{P} + vh\mathbf{d}$, $0\le v \le 1$, то мы получим поверхность выдавливания по направлению $\mathbf{d}$. Эта поверхность будет описываться следующей формулой:
\begin{equation}
    \mathbf{S}(u,v) = \mathbf{C}(u)+vh\mathbf{d}
\end{equation}

Тогда, например, если в качестве кривой $\mathbf{C}(u)$ взять NURBS окружность (Рисунок \ref{img:mycircle}), то можно получить поверхность выдавливания изображенную на Рисунке \ref{img:extrude}.

\includeimage
{mycircle} % Имя файла без расширения (файл должен быть расположен в директории inc/img/)
{f} % Обтекание (без обтекания)
{h} % Положение рисунка (см. figure из пакета float)
{0.4\textwidth} % Ширина рисунка
{Окружность, выполненная с помощью NURBS} % Подпись рисунка
\includeimage
{extrude} % Имя файла без расширения (файл должен быть расположен в директории inc/img/)
{f} % Обтекание (без обтекания)
{h} % Положение рисунка (см. figure из пакета float)
{0.4\textwidth} % Ширина рисунка
{Поверхность выдавливания} % Подпись рисунка


\newpage
\section{Поверхность вращения}
Пусть кривая $\mathbf{C}(u)$ задана в плоскости $XZ$. Для создания поверхности вращения повернём эту кривую вокруг оси $z$, изменяя параметр $v$, где $v$ определяет угол, под которым каждая точка повернута относительно оси. И пусть точка $\mathbf{P_s} = \{x_s, 0,0\}$ - задает положения оси вращения в локальной системе координат. Тогда поверхность вращения можно получить с помощью следующей формулы:
\begin{equation}
    \mathbf{S}(u,v) = \mathbf{P_s}+\mathbf{M}(v)(\mathbf{C}(u)-\mathbf{P_s}),
\end{equation}
где 
\begin{equation}
    \mathbf{M}(v) = \left(
        \begin{matrix}
            \cos{v} & 0 & 0\\
            \sin{v} & 0 & 0\\
            0 & 0 & 1
        \end{matrix}
    \right)
\end{equation}

Так, например, для NURBS окружность построение поверхности вращения и резуьтат будет выглядить следующим образом:
\includeimage
{rot1} % Имя файла без расширения (файл должен быть расположен в директории inc/img/)
{f} % Обтекание (без обтекания)
{h} % Положение рисунка (см. figure из пакета float)
{0.5\textwidth} % Ширина рисунка
{Пример построения поверхности вращения} % Подпись рисунка

\includeimage
{rot2} % Имя файла без расширения (файл должен быть расположен в директории inc/img/)
{f} % Обтекание (без обтекания)
{h} % Положение рисунка (см. figure из пакета float)
{0.5\textwidth} % Ширина рисунка
{Поверхность вращения} % Подпись рисунка

\section{Поверхность сдвига}
В общем случае кривая $\mathbf{g}(v)$ не обязательно задает прямую. Тогда с помощью формулы
\begin{equation}
    \mathbf{S}(u,v)=\mathbf{g}(v) + (\mathbf{C}(u)-\mathbf{g}(v_{\min}))
\end{equation}
можно получить поверхность сдвига.

Для направляющей кривой, изображенной на Рисунке \ref{img:dimsp}, и образующей NURBS окружности получим поверхность сдвига изображенную на Рисунке \ref{img:shift}.

\includeimage
{dimsp} % Имя файла без расширения (файл должен быть расположен в директории inc/img/)
{f} % Обтекание (без обтекания)
{h} % Положение рисунка (см. figure из пакета float)
{0.2\textwidth} % Ширина рисунка
{Направляющий сплайн} % Подпись рисунка

\includeimage
{shift} % Имя файла без расширения (файл должен быть расположен в директории inc/img/)
{f} % Обтекание (без обтекания)
{h} % Положение рисунка (см. figure из пакета float)
{0.2\textwidth} % Ширина рисунка
{Поверхность сдвига} % Подпись рисунка

\section{Кинематические поверхности}
Для построения кинематической поверхности необходимо вычислить подвижную декартову систему координат.
Первый базисный вектор подвижной СК $\mathbf{i}_1= \frac{\mathbf{g}'}{|\mathbf{g}'|}$ направим по касательной к направляющей кривой. Второй $\mathbf{i}_2$ направим ортогонально первому, а $\mathbf{i}_3 = \mathbf{i}_1\times\mathbf{i}_2$.

Тогда для вычисления радиус-вектора точки кинематической поверхности построим матрицу
\begin{equation}
    \mathbf{A}(v) = [\mathbf{i}_1(v)~~~\mathbf{i}_2(v)~~~\mathbf{i}_3(v)].
\end{equation}

Матрица $\mathbf{A}(v)$ является матрицей преобразования координат радиус-вектора точки из подвижной СК в глобальную и зависит от параметра направляющей кривой.

Запомним положение образующей кривой $\mathbf{C}(u)$ в подвижном касательном базисе в начале направляющей и будем сохранять его при движении вдоль напраляющей. Тогда радиус-вектор точки образующей в подвижной СК при $v=v_{\min}$ равен
\begin{equation}
    \mathbf{X}(u, v_{\min}) = \mathbf{A}^{-1}(v_{\min})\cdot(\mathbf{C}(u)-\mathbf{g}(v_{\min})).
\end{equation}

При движении вдоль направляющей кривой подвижный касательный базис меняет свое положение и ориентацию в пространстве и увлекает за собой жестко связанную с ним образующую кривую. Вектор $\mathbf{X}(u, v_{\min})$ выражает положение точки образующей относительно точки на навравляющей кривой в подвижном базисе, которое сохраняется для произвольного параметра $v$. Переходя из подвижной СК в глобальную при текущем параметре $v$, получим радиус-вектор точки на кинематической поверхности
\begin{equation}
    \mathbf{S}(u,v) = \mathbf{g}(v) + \mathbf{A}(v)\cdot\mathbf{X}(u, v_{\min}).
\end{equation} 

Таким образом, радиус вектор кинематической поверхности опишем следующей функцией
\begin{equation}
    \mathbf{S}(u,v) = \mathbf{g}(v) + \mathbf{M}(v)\cdot(\mathbf{C}(u)-\mathbf{g}(v_{\min}))
\end{equation}
где $\mathbf{M}(v)$ - матрица поворота текущего подвижного базиса относительно его начального положения. Эта матрица вычисляется по формуле
\begin{equation}
    \mathbf{M}(v) = \mathbf{A}(v)\cdot\mathbf{A}^{-1}(v_{\min})
\end{equation}

Так, например, для направляющей, изображенной на Рисунке \ref{img:dimsp2}, и образующей NURBS окружности получим кинематическую поверхность, изображенную на Рисунке \ref{img:sweep}.

\includeimage
{dimsp2} % Имя файла без расширения (файл должен быть расположен в директории inc/img/)
{f} % Обтекание (без обтекания)
{h} % Положение рисунка (см. figure из пакета float)
{0.5\textwidth} % Ширина рисунка
{Направляющий сплайн} % Подпись рисунка

\includeimage
{sweep} % Имя файла без расширения (файл должен быть расположен в директории inc/img/)
{f} % Обтекание (без обтекания)
{h} % Положение рисунка (см. figure из пакета float)
{0.5\textwidth} % Ширина рисунка
{Поверхность сдвига} % Подпись рисунка

\chapter{Программная реализация}

В рамках курсовой работы была выполнена программная реализация выше изложенных методов на языке C++. Она состоит из графической оболочки, поддерживающую визуализацию поверхностей и кривых, а также инструментов для создания кривых и поверхностей движения. Отрисовка двух и трехмерных объектов на экране осуществлена с помощью OpenGL.

Таким образом, программа разбита на классы отвечающие за отрисовку (Renderer) и классы осуществляющие математическую реализацию кривых и поверхностей.

\section{Графическая оболочка}

Графическая оболочка написана с использованием библиотеки c открытым исходным кодом для создания пользовательских интерфейсов Dear ImGui, предоставляющая отрисовку интерфейса в фреймбуффер OpenGL. Общий вид графической оболочки можно увидеть на Рисунке \ref{img:gobl}.

\includeimage
{gobl} % Имя файла без расширения (файл должен быть расположен в директории inc/img/)
{f} % Обтекание (без обтекания)
{h} % Положение рисунка (см. figure из пакета float)
{0.5\textwidth} % Ширина рисунка
{Графическая оболочка} % Подпись рисунка

Пользовательский интерфейс состоит из окон: инструментов (в верхней части экрана), загруженных кривых (Sketches), кривых, определенных в рабочем пространстве (DimSplines), свойств выбранного объекта на сцене (Properties). А также окна рабочего пространства (Viewport) и окна редактирования кривых (Sketch Editor).

В окне редактирования кривых можно создавать или редактировать кривые (Рисунок \ref{img:sedit}). В свою очередь тип кривой можно выбрать в окне Properties, как показано на Рисунке \ref{img:prop}.

\includeimage
{sedit} % Имя файла без расширения (файл должен быть расположен в директории inc/img/)
{f} % Обтекание (без обтекания)
{h} % Положение рисунка (см. figure из пакета float)
{0.5\textwidth} % Ширина рисунка
{Окно Sketch Editor} % Подпись рисунка\

\includeimage
{prop} % Имя файла без расширения (файл должен быть расположен в директории inc/img/)
{f} % Обтекание (без обтекания)
{h} % Положение рисунка (см. figure из пакета float)
{0.3\textwidth} % Ширина рисунка
{Окно Properties} % Подпись рисунка\

\section{Инструменты создания поверхностей движения}

Перед созданием поверхности движения нужно создать или загрузить готовую обрузующую кривую в окнах Sketch и Sketch Editor. Затем если необходимо создать опорную плоскость, с помощью соответствующего инструмента, а затем выбрать инструмент создания требуемой поверхности движения.

Рассмотрим пример работы с программой для создания поверхостей выдавливания и вращения.

\subsection{Поверхность выдавливания}

После создания опорной плоскости, прикрепления к ней нужной образующей кривой и выбрав инструмент создания поверхности выдавливания (Extrude), в окне Viewport программы отобразится вспомогательные элементы для создания поверности выдавливания (Рисунок \ref{img:extr}). Также в окне Extrude можно задать свойства поверхности выдавливания: направление выдавливания и длину (Рисунок \ref{img:exprop}). При нажатии кнопки Create будет создана поверхность с выбранными свойствами, как, например, на Рисунке (\ref{img:extrude}).

\includeimage
{extr} % Имя файла без расширения (файл должен быть расположен в директории inc/img/)
{f} % Обтекание (без обтекания)
{h} % Положение рисунка (см. figure из пакета float)
{0.3\textwidth} % Ширина рисунка
{Инструмент создания поверхности выдавливания} % Подпись рисунка\

\includeimage
{exprop} % Имя файла без расширения (файл должен быть расположен в директории inc/img/)
{f} % Обтекание (без обтекания)
{h} % Положение рисунка (см. figure из пакета float)
{0.3\textwidth} % Ширина рисунка
{Свойства поверхности выдавливания} % Подпись рисунка\

\subsection{Поверхность вращения}

Аналогичным образом, после создания опорной плоскости, образующей кривой и выбора инструмента создания поверхности вращения отобразится вспомогательная графика - ось вращения (Рисунок \ref{img:rot3}), а также в окне Rotate отобразятся свойства создаваемой поверхности: радиус и угол вращения.

\includeimage
{rot3} % Имя файла без расширения (файл должен быть расположен в директории inc/img/)
{f} % Обтекание (без обтекания)
{h} % Положение рисунка (см. figure из пакета float)
{0.3\textwidth} % Ширина рисунка
{Инструмент создания поверхности вращения} % Подпись рисунка\


\chapter*{ЗАКЛЮЧЕНИЕ}
\addcontentsline{toc}{chapter}{ЗАКЛЮЧЕНИЕ}

В рамках курсовой работы были рассмотрены методы построения поверхностей и кривых, а именно  кривые и поверхности Безье, рациональные кривые и поверхности Безье, B-Spline и NURBS. А также алгоритмы для их построения на компьютере.

Затем были рассмотрены методы построения поверхностей движения, таких как поверхности выдавливания, сдвига, вращения и кинематические поверхности.

В рамках курсовой работы была выполнена программная реализация выше изложенных методов, которая поддерживает создание и редактирование кривых в реальном времени, проектирование поверхностей движения с различными параметрами, а также визуализацию полученных математических объектов на экране.

\makebibliography

\begin{appendices}
	\chapter{Алгоритмы}
\includelistingpretty
{ders.txt} % Имя файла с расширением (файл должен быть расположен в директории inc/lst/)
{c++} % Язык программирования (необязательный аргумент)
{Алгоритм вычисления производных базисных функций на языке C++} % Подпись листинга

\end{appendices}

\end{document}